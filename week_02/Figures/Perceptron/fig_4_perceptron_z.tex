% create a file name latexmkrc and copy this code on it:
% END { system('convert -density 300 output.pdf myImage.png'); }

\documentclass[convert={density=300,size=1080x800,outext=.png}]{standalone}
\usepackage{neuralnetwork}
\usepackage{tikz}
\usepackage{xpatch}
\makeatletter

% source: https://tex.stackexchange.com/questions/406167/generalize-neural-network
% \linklayers have \nn@lastnode instead of \lastnode,
% patch it to replace the former with the latter, and similar for thisnode
\xpatchcmd{\linklayers}{\nn@lastnode}{\lastnode}{}{}
\xpatchcmd{\linklayers}{\nn@thisnode}{\thisnode}{}{}
\makeatother

\begin{document}

\begin{neuralnetwork}[height=1,maintitleheight=1cm,layertitleheight=2.5cm, layerspacing=50]
  \newcommand{\textinput}[2]{\ifnum #2=4 $x_n$ \else $x_#2$ \fi}
  \newcommand{\textactivation}[2]{$g(.)$}
  \newcommand{\textsum}[2]{$\sum$}
  \newcommand{\textout}[2]{$z$}
  
  \newcommand{\w}[4] {\ifnum #2=4 $w_n$ \else $w_#2$ \fi}
  \newcommand{\ws}[4] {}
  \setdefaultlinklabel{\w}

  \inputlayer[count=4, title={}, bias=True, exclude={3}, text=\textinput]
  \outputlayer[count=1, title={}, bias=False, text=\textout]
  \linklayers[not from={3}]
  
  % draw dots
    \path (L0-2) -- node{$\vdots$} (L0-4);
    \draw [->,>=stealth,gray] (L1-1) -- node [black,midway,above] {$\hat{y}=g(z)$} +(1.6,0);
\end{neuralnetwork}

\end{document}
